\documentclass[../main.tex]{subfiles}

\begin{document}

\قسمت{از تقاضا به \متن‌لاتین{URL}}

\پاراگراف{}
دوست شما تقاضای \متن‌لاتین{HTTP} زیر را برای شما ارسال کرده است.
آیا می‌توانید مشخص کنید که \متن‌لاتین{URL} ای که در مرورگر وارد شده است چه مقداری داشته است؟
چه قسمتی از \متن‌لاتین{URL} قابل مشخص کردن نیست؟

\begin{latin}
\begin{minted}[bgcolor=LightGray]{http}
GET /hello?name=parham HTTP/1.1
Host: httpbin.org
Connection: keep-alive
User-Agent: curl/7.81.0
\end{minted}
\end{latin}

رویه‌ای که از زمان وارد کردن \متن‌لاتین{URL} در مرورگر شما تا نمایش صفحه اتفاق می‌افتد (مانند استفاده از \متن‌لاتین{DNS}، شکل‌گیری ارتباط \متن‌لاتین{TCP} و \نقاط‌خ) را توضیح دهید.

\پاراگراف{}
۳ نمره

\begin{answer}

قسمت غیرقابل مشخص کردن مربوط به \متن‌لاتین{Fragment} است که تنها سمت کلاینت مورد استفاده قرار می‌گیرد.

\شروع{شمارش}
\فقره قمست \متن‌لاتین{Host} از \متن‌لاتین{URL} وارد شده با استفاده از \متن‌لاتین{DNS} به \متن‌لاتین{IP} تبدیل می‌شود.
\فقره یک ارتباط \متن‌لاتین{TCP} با \متن‌لاتین{IP} بدست آمده و پورت مشخص شده در \متن‌لاتین{URL} ساخته می‌شود.
\فقره درخواست \متن‌لاتین{HTTP} بر روی این ارتباط \متن‌لاتین{TCP} ساخته شده، ارسال می‌گردد.
\فقره سرور درخواست را با توجه به پارامترهای متد، مسیر و کوئری پردازش کرده و پاسخ می‌دهد.
\فقره پاسخ روی ارتباط \متن‌لاتین{TCP} بازگردانده می‌شود.
\پایان{شمارش}

\end{answer}

\end{document}
