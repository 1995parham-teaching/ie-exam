\documentclass[../main.tex]{subfiles}

\begin{document}
\قسمت{آقای یا خانم؟}

\زیرقسمت{مقدمه}
\پاراگراف{}
یکی از سخترین کارها در کلاس و مخصوصا کلاس‌های آنلاین استفاده صحیح از پیشوندهای آقای و خانم می‌باشد.
مثلا شما در لیست اسم پرهام الوانی را می‌بینید و نمی‌دانید باید بگویید آقای الوانی یا خانم الوانی.
در این تمرین قصد داریم وب‌سایتی طراحی کنیم که در این امر مهم اساتید را یاری دهد.

\زیرقسمت{وبگاه}
\پاراگراف{}
این وبگاه از شِمای زیر پیروی می‌کند. یک پس زمینه تمام صفحه را در بر گرفته است و در میان آن یک مستطیل شفاف قرار گرفته است.
دقت داشته باشید شفافیت نباید به قدری باشد که متن‌های درون مستطیل خوانایی نداشته باشند.

\begin{figure}[h]
  \centering
  \includegraphics[scale=0.25]{./genderize-top-level}
  \caption{طراحی سطح بالا}
\end{figure}

\پاراگراف{}
مستطیل میانی تمامی محتویات قابل نمایش شما را می‌بایست شامل شود. این مستطیل می‌بایست تنها به اندازه محتویات باشد اما برای نمایش زیباتر آن از لایه‌گذاری\پانویس{padding} استفاده کنید.
محتوای شما از دو قسمت تشکیل شده است که به صورت افقی کنار یکدیگر قرار گرفته‌اند. قسمت اول یک فرم است که خود از دو ورودی تشکیل شده است. وروردی اول به شکل متنی بوده و کاربر اسم مورد نظر را وارد می‌کند.
ورودی دوم به شکل دو دکمه رادیویی است که به کاربر می‌کند اگر از پیش جواب را می‌داند آن را وارد کرده و این جواب برای آینده ذخیره می‌شود. در مورد جواب‌های ذخیره شده در ادامه بیشتر صحبت خواهد شد.

\begin{figure}[h]
  \centering
  \includegraphics[scale=0.3]{./genderize-content}
  \caption{طراحی مستطیل محتوا}
\end{figure}

\پاراگراف{}
پس از پر کردن فرم کاربر گزینه درخواست\پانویس{submit} را زده و به این درخواست ایشان ارسال می‌شود. برای پیش‌بینی جنسیت اسم داده شده از وبگاه زیر استفاده می‌کنیم:

\begin{latin}\begin{center}
https://api.genderize.io/?name=hassan
\end{center}\end{latin}

شما نام را در قالب یک رشته تقاضا\پانویس{query string} و به صورت \متن‌لاتین{GET} برای این وبگاه ارسال می‌کنید و در نهایت حاصل پیش‌بینی را به کاربر نمایش می‌دهید. خروجی این درخواست به شکل زیر می‌باشد:

\begin{latin}
\begin{minted}[bgcolor=LightGray]{json}
{
  "name":"hassan",
  "gender":"male",
  "probability":0.97,
  "count":49197
}
\end{minted}
\end{latin}

\شروع{امتیازی}
ممکن است به دلیل شبکه ارسال درخواست به خطا بخورد یا اینکه این وبگاه نتواند برای یک نام پیش‌بینی داشته باشد (برای نمونه این وبگاه برای نام ``محمد حسین'' پیش‌بینی ندارد). شما می‌بایست این خطاها را با نمایش پیام خطا به کاربر رسیدگی کنید. دقت داشته باشید که نمایش می‌بایست در وبگاه شما صورت بگیرد و نمایش در کنسول مرورگر نمره‌ای دریافت نمی‌کند.
\پایان{امتیازی}

\زیرقسمت{جواب‌های ذخیره شده}
زمان‌هایی وجود دارد که وبگاه پیش‌بینی اشتباه انجام می‌دهد، در این صورت می‌توان با پرسش از دانشجو به پاسخ صحیح رسید. در این صورت این سامانه می‌بایست توانایی ذخیره کردن پاسخ صحیح را داشته باشد. برای ذخیره کردن پاسخ صحیح یک راه استفاده از کوکی‌ها می‌باشد. در نهایت اگر اسم وارد شده در قالب کوکی موجود باشد شما می‌بایست پاسخ ذخیره شده را نیز به کاربر نمایش دهید.
در صورتی که یک ذخیره‌سازی برای یک نام صورت گرفته باشد و کاربر دوباره از قسمت ذخیره‌سازی استفاده کند مقدار قبلی \متن‌سیاه{پاک} خواهد شد. در صورتی که کاربر بخواهد مقدار ذخیره شده برای یک اسم را پاک کند، بعد از ارسال درخواست برای آن نام می‌تواند از دکمه پاک کردن که در قسمت جواب‌های ذخیره شده نمایش داده می‌شود استفاده کند. دقت داشته باشید که قسمت جواب‌های ذخیره شده
در صورت وجود جواب ذخیره شده به نمایش درخواهد آمد.

\شروع{امتیازی}
می‌توانید به جای استفاده از کوکی‌ها از \متن‌لاتین{local storage} استفاده کنید. برای آشنایی بیشتر می‌توانید از وبگاه زیر کمک بگیرید.


\begin{latin}\begin{center}
https://developer.mozilla.org/en-US/docs/Web/API/Window/localStorage
\end{center}\end{latin}

\پایان{امتیازی}

\زیرقسمت{نکات پیاده‌سازی}

\شروع{فقرات}
    \فقره برای کدهایتان از کامنت استفاده کنید. توضیح کارکرد بلاک‌های \متن‌لاتین{CSS} اجباری می‌باشد. توابعی و قطعات کد جاوا اسکریپت نیز می‌بایست حداقل یک خط کامنت داشته باشند.
    \فقره کامنت فارسی یا انگلیسی موردی ندارد اما از فینگلیش (!) نوشتن پرهیز کنید.
    \فقره استفاده از کتابخانه‌ها و چهارچوب‌ها در پروژه مجاز \متن‌سیاه{نمی‌باشد}.
    \فقره از آنجایی که این پروژه در قالب \متن‌سیاه{امتحان میانترم} می‌باشد از تغییر دادن صورت مساله یا انجام موارد خارج از موارد مطرح شده خودداری کنید.
\پایان{فقرات}

\زیرقسمت{پرسش‌های متداول}

صورت پروژه در مهلت انجام پروژه بر پایه سوالات شما به روزرسانی خواهد شد.
\شروع{توضیح}
\فقره[پرسش (۱)] اگر کاربر دو دیتای متفاوت برای یک اسم ذخیره کند تکلیف چیست؟
\فقره[پاسخ] آخرین مقدار بر روی مقدارهای قبلی ذخیره می‌شود.
\فقره[پرسش (۲)] عکس پس‌زمینه خاصی مدنظر است؟ اگر نیست قرار دادن آن اجباری است؟ \متن‌لاتین{CSS} خاصی باید داشته باشد؟
\فقره[پاسخ] خیر عکس خاصی مدنظر نیست، اما قرار دادن عکس اجباری می‌باشد. دقت داشته باشید که این عکس می‌بایست تمام پس‌زمینه را پوشانده باشد و در صفحه‌های نمایش متفاوت نمایش صحیح داشته باشد. لازم به ذکر است نیازی نیست وبگاه شما قابل نمایش در گوشی‌های موبایل و \نقاط باشد بلکه تنها کافی است در صفحات نمایش دسکتاپ به صورت مناسب نمایش داده شود.
\فقره[پرسش (۳)] اگر کاربر جنسیت رو مشخص کند، درخواست نباید ارسال شود؟
\فقره[پاسخ] درخواست می‌بایست همیشه ارسال شود تا بتوان آنچه کاربر ذخیره کرده یا می‌کند با پیش‌بینی سایت مطابقت داد.
\فقره[پرسش (۴)] آیا استفاده از کتابخانه‌ی \متن‌لاتین{bootstrap} مجاز است؟
\فقره[پاسخ] خیر مجاز \متن‌سیاه{نمی‌باشد}.
\فقره[پرسش (۵)] برای قسمت امتیازی که گفته شده است نمایش خطاها در وبگاه صورت بگیرد، اینکه صرفا یک \متن‌لاتین{alert} بدهد مورد قبول است؟
\فقره[پاسخ] خیر، نمایش می‌بایست در قالب ارائه یک متن در داخل وب‌گاه صورت بپذیرد و استفاده از \متن‌لاتین{alert} و \متن‌لاتین{confirm} مجاز نمی‌باشد.
\فقره[پرسش (۶)] آیا برای نمایش خطاها در صفحه، تغییر شِما ایرادی ندارد؟
\فقره[پاسخ] خیر برای این قسمت \متن‌سیاه{امتیازی} می‌توانید مقداری تغییر داشته باشید.
\پایان{توضیح}

\end{document}
