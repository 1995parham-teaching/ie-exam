\documentclass[../main.tex]{subfiles}

\begin{document}

\قسمت{فرق میان \متن‌لاتین{Slice} و \متن‌لاتین{Array}}

\زیرقسمت{تغییر کردن یا نکردن، مساله این است!}
\پاراگراف{}
کد زیر را در نظر بگیرید، آیا \متن‌لاتین{arr} قبل و بعد از فراخوانی \متن‌لاتین{changeMe} تغییر کرده است؟ توضیح دهید.

\begin{latin}
\begin{minted}[bgcolor=LightGray]{go}
package main

import "fmt"

func changeMe(arr [5]int) {
        arr[0] = 10
}

func main() {
        arr := [5]int{1, 2, 3, 4, 5}

        fmt.Printf("arr: %+v\n", arr)
        changeMe(arr)
        fmt.Printf("arr: %+v\n", arr)
}
\end{minted}
\end{latin}

\زیرقسمت{بهینه‌سازی}
\پاراگراف{}
یکی از هم تیمی‌های شما کد زیر را به بهانه بهینه‌سازی نوشته است که هدف
از آن پر کردن \متن‌لاتین{arr} از خانه ۰ تا ۹ با مقدارهای ۰ تا ۹ است.
آیا این کد عملکرد صحیحی دارد؟ دلیل خود را توضیح دهید و در صورتی
که کد مشکلی دارد، آن را اصلاح کنید.

\begin{latin}
\begin{minted}[bgcolor=LightGray]{go}
package main

import "fmt"

func main() {
  arr := make([]int, 10)

    for i := 0; i < 10; i++ {
      arr = append(arr, i)
    }

    fmt.Printf("it must be 2: %d\n", arr[3])
}

\end{minted}
\end{latin}

\پاراگراف{}
۲ نمره

\begin{answer}

هر قسمت ۵۰ درصد نمره را دارا است.

\متن‌سیاه{قسمت اول}.
به دلیل استفاده از آرایه (اشاره به آرایه ۲۵ درصد نمره است) و کپی شدن آن (یا فراخوانی آن با مقدار)، تغییری در \متن‌لاتین{arr} صورت نمی‌گیرد.

\متن‌سیاه{قسمت دوم}.
پارامتر اول در ساخت \متن‌لاتین{slice} تعداد خانه‌های آن است (اشاره به توصیف این پارامتر ۲۵ درصد نمره است) بنابراین این کد یک \متن‌لاتین{slice} با ۱۰ خانه ساخته و در ادامه ۱۰ خانه دیگر به آن اضافه می‌کند بنابراین
کد اشتباه است.

\end{answer}

\end{document}
