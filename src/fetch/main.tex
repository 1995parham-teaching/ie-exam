\قسمت{پرتال امن}

دانشگاه قصد دارد سرویس جدیدی ارائه دهد.
این سرویس یک قسمت وب دارد که با استفاده از جاوا اسکریپت توسعه پیدا
کرده است و در مرورگر کاربر اجرا می‌شود.

دانشگاه قصد دارد از تابع \متن‌لاتین{fetch} برای دسترسی به پرتال دانشکده استفاده کند.
در این کاربرد تقاضا با نام کاربری و رمز عبور \متن‌سیاه{ادمین} پرتال ارسال می‌شود و اطلاعات دانشجو برای نمایش گرفته می‌شود.
آیا از نظر شما استفاده از این شیوه امن می‌باشد؟ فرض کنید دانشگاه از بستر ارتباطی \متن‌لاتین{HTTPS} استفاده می‌کند و
کاربران این سامانه دانشجویان هستند و شناختی بر آن‌ها وجود \متن‌سیاه{ندارد}.
توضیح دهید.

۲ نمره

\شروع{پاسخ}

در این روش به دلیل \متن‌لاتین{HTTPS} شنودی رخ نمی‌دهد.
اما مشکل اینجاست که وقتی کاربران با نام کاربری و رمز عبور ادمین تقاضا ارسال می‌کنند یعنی این مقادیر را دارند. یعنی به رمز عبور ادمین دسترسی دارند که درون فایل \متن‌لاتین{Webpage}شان قرار دارد.
همچنین دانشجویان نباید سطح دسترسی ادمین داشته باشند. پس کلاً نباید احراز هویت آن‌ها با مشخصات ادمین صورت پذیرد.

\شروع{فقرات}
\فقره عدم بیان دسترسی کاربران به اطلاعات ادمین پرتال: از دست دادن ۲۰ درصد نمره
\فقره امن بودن ارتباط ارتباطی با استفاده از تابع \متن‌لاتین{fetch} ندارد: از دست دادن ۳۰ درصد نمره
\فقره واضح است که استفاده از اطلاعات مهم سمت کلاینت امنیت ندارد اما می‌بایست بیان شود که علت این امر چیست.
\فقره با توجه به اینکه در صورت سوال بر عدم شناخت کاربران تاکید شده است، این مورد می‌بایست در پاسخ دخیل شده باشد.
\پایان{فقرات}

\پایان{پاسخ}
