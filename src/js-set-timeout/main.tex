\documentclass[../main.tex]{subfiles}

\begin{document}

\قسمت{تعیین مهلت}

\پاراگراف{}
سپهر کد زیر را نوشته است و برایش سوال ایجاد شده است که چرا با وجود استفاده از زمان صفر به عنوان مقدار \متن‌لاتین{Timeout}
هنوز تابعی که به عنوان \متن‌لاتین{callback} داده شده است، در انتها اجرا میشود و در محلی که نوشته شده است فراخوانی نشده است. آیا می‌توانید به او کمک کنید و مساله را برایش توضیح دهید؟
آیا این مساله تنها روی سیستم سپهر رخ می دهد؟

\begin{latin}
\begin{minted}[bgcolor=LightGray]{js}
(function () {
  let el = document.getElementById("sepehr-awesome-code");

  setTimeout(function cb() {
    el.innerHTML += "I am the callback<br />";
  }); // has a default time value of 0

  el.innerHTML += "I am after the callback! <br />";
})();
\end{minted}
\end{latin}

\پاراگراف{}
۳ نمره

\begin{answer}

در جاوا اسکریپت و در تابع \متن‌لاتین{setTimeout} زمان صفر به معنی این نیست که تابع مورد نظر بدون هیچ زمان انتظاری فوراً اجرا می‌شود.
یعنی فراخوانی یک \متن‌لاتین{Callback} با این تابع و زمان صفر لزوماً به معنی اجرای آن Callback بدون وقفه نیست.

در جاوا اسکریپت مدل اجرای توابع و کارها به صورت \متن‌لاتین{Event loop} است. به این صورت که در یک صفی تسک‌ها قرار می‌گیرند و یک حلقه می‌آید کارها را از روی آن برداشته و آن‌ها را به ترتیب اجرا می‌کند.
پس در هر لحظه فقط یک کار انجام می‌شود.
حال در کد سپهر با فراخوانی تابع \متن‌لاتین{setTimeout} و سپردن یک تسک به آن، این تسک در صف قرار گرفته و پس از اتمام تسک فعلی که کل کد داده شده در سؤال (به جز تابع درون \متن‌لاتین{setTimeout}) است این تسک اجرا می‌شود.

\end{answer}

\end{document}
