\قسمت{\متن‌لاتین{As A Service}}

\زیرقسمت{برنامه سخت}

شما قصد دارید برای استقرار یک برنامه از مجازی‌سازی استفاده کنید، این برنامه را خودتان توسعه داده‌اید و برای اجرا نیازمند تنظیمات در سطح سیستم عامل است.
از بین \متن‌لاتین{IaaS}، \متن‌لاتین{PaaS} و \متن‌لاتین{SaaS} کدام را پیشنهاد می‌کنید؟ دلیل خود را توضیح دهید.

\زیرقسمت{برنامه راحت}

یک سازمان قصد دارد یک سیستم حضور و غیاب بالا بیاورد و می‌خواهد کمترین درگیری را برای مدیریت آن داشته باشد و از برنامه‌ای آماده استفاده کند.
از بین \متن‌لاتین{IaaS}، \متن‌لاتین{PaaS} و \متن‌لاتین{SaaS} کدام را پیشنهاد می‌کنید؟ دلیل خود را توضیح دهید.


۲ نمره

\شروع{پاسخ}

این پاسخ برگفته از پاسخ خانم مریم علی کرمی به شماره دانشجویی ۹۷۳۱۰۴۵ است.

برنامه سخت: برای این کار بهتر از \متن‌لاتین{IaaS} که \متن‌لاتین{Infrastructure} را به عنوان سرویس در اختیار ما می‌گذارد، استفاده کنیم
چرا که منابع \متن‌لاتین{Computing} را با قابلیت‌های \متن‌لاتین{storage} و \متن‌لاتین{networking} در اختیار ما قرار می‌دهد.
این منابع \متن‌لاتین{Scalable} هستند و بنابه استفاده افزایش پیدا می‌کنند.
از آنجایی که ما برنامه خود را به صورت آماده داریم و به قابلیت های \متن‌لاتین{OS} نیاز داریم این سرویس برای ما مناسب است.

برنامه راحت: در این حالت بهتر است که از \متن‌لاتین{SaaS} یا \متن‌لاتین{Software as a service} استفاده کنیم چرا که می‌خواهیم
کمترین میزان درگیری را داشته باشیم و برنامه را خودمان پیاده سازی نکنیم.
در این حالت \متن‌لاتین{SaaS} می‌تواند نیاز ما را با ارائه برنامه‌ای که
\متن‌لاتین{host} شده و برای دیگران قابل استفاده است برطرف نماید. کافی است که ما \متن‌لاتین{Adminstrator} آن
باشیم.

\پایان{پاسخ}
