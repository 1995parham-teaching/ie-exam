\قسمت{چرا \متن‌لاتین{Thrift}، \متن‌لاتین{RPC} است؟}

بیان شد که \متن‌لاتین{Thrift} یک چهارچوب \متن‌لاتین{RPC}\پانویس{Remote Procedure Call} است، این موضوع به چه معناست؟
آیا \متن‌لاتین{JSON} به تنهایی یک چهارچوب \متن‌لاتین{RPC} محسوب می‌شود؟

۲ نمره

\شروع{پاسخ}

\متن‌لاتین{Thrift} این امکان را می‌دهد که اینترفیس سرویس مورد نظر خودمان را درون یک تکه فایل ساده تعریف کنیم.
کامپایلر \متن‌لاتین{Thrift} با گرفتن این فایل به عنوان ورودی، در خروجی کدی را تولید میکند که بتوان از آن در سمت سرور و کلاینت استفاده کرد.
به این صورت که یک سرویس یا تابعی درون سرور از طرف کلاینت فراخوانی می‌شود. پس یعنی یک سرویس \متن‌لاتین{RPC} را به کمک یه فایل ساده فراهم می‌کند.
به همین دلیل یک \متن‌لاتین{Framework} محسوب می‌شود، یک \متن‌لاتین{RPC Framework}.

\متن‌لاتین{JSON} نه یک چهارچوب است و نه یک سرویس \متن‌لاتین{RPC}. بلکه فقط برای تبادل اطلاعات بین کلاینت و سرور استفاده می‌شود. اما \متن‌لاتین{JSON-RPC} را داریم که یک پروتکل استفاده از \متن‌لاتین{RPC}
بر پایه \متن‌لاتین{JSON} است.

تعریف \متن‌لاتین{RPC} دو سوم نمره را داشته و یک سوم آن به اشاره صحیح موضوع \متن‌لاتین{RPC} نبودن \متن‌لاتین{JSON} به تنهایی، تعلق می‌گیرد.

\شروع{پاسخ}

\end{document}
