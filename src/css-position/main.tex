\قسمت{مکان‌یابی در \متن‌لاتین{CSS}}

در رابطه با انواع جایگیری‌ها در \متن‌لاتین{CSS} توضیح دهید. برای هر جایگیری یک تعریف کوتاه در حد یک خط کفایت می‌کند. مشخص کنید در کدام
جایگیری‌ها فضایی در محل پیش‌فرض المان لحاظ نمی‌گردد؟ کدام جایگیری در واقع ترکیبی از دو جایگیری دیگر است؟

۲ نمره

\شروع{پاسخ}

\شروع{فقرات}

\فقره \متن‌سیاه{\متن‌لاتین{static}}:
حالت پیشفرض جایگیری در \متن‌لاتین{css} است که هیچ کدام از مقادیر
\متن‌لاتین{top}، \متن‌لاتین{bottom}، \متن‌لاتین{right} و \متن‌لاتین{left} روی آن تاثیری ندارند.
این المنت‌ها هیچ حالت جایگیری خاصی ندارند و در همان جایی از صفحه که قرار بوده باشند قرار می‌گیرند. در این
حالت المنت در فلوی صفحه قرار دارد.

\فقره \متن‌سیاه{\متن‌لاتین{relative}}:
حالتی از جایگیری است که المنت نسبت به مکان نرمال و اولیه خودش جابه‌جا شود.در این حالت المنت در فلوی
صفحه قرار دارد.

\فقره \متن‌سیاه{\متن‌لاتین{absolute}}:
موقعیت این المنت نسبت به نزدیک ترین پدرش که جایگیری \متن‌لاتین{static} نداشته باشد تعیین می‌شود.
اگر جایگیری همه
المنت‌ها \متن‌لاتین{Static} باشد این المنت نسبت به \متن‌لاتین{body} جای‌گیری می‌کند
و با اسکرول کردن پیج جابه جا میشود.در این
حالت المنت از فلوی صفحه خارج می‌شود.

\فقره \متن‌سیاه{\متن‌لاتین{fixed}}:
موقعیت این المنت نسبت به \متن‌لاتین{viewport} تایین می شود یعنی در صورتی که صفحه اسکرول بخورد یا جابه‌جا شود نیز
در جایی که با
\متن‌لاتین{top}،
\متن‌لاتین{bottom}،
\متن‌لاتین{right} و
\متن‌لاتین{left}
برای آن مشخص کرده‌ایم باقی می‌ماند. در این حالت المنت از فلوی صفحه
خارج می‌شود.

\فقره \متن‌سیاه{\متن‌لاتین{sticky}}:
این موقعیت ترکیبی از \متن‌لاتین{relative} و \متن‌لاتین{fixed} است. اول به صورت \متن‌لاتین{relative} است تا اینکه تا یک حد خاصی \متن‌لاتین{scroll} کنیم
و از آنجا به بعد به صورت \متن‌لاتین{fixed} جایگیری می کند.در این حالت المنت از فلوی صفحه خارج نمی‌شود.

\پایان{فقرات}

در اینجا \متن‌لاتین{flow} به معنای محل پیش‌فرض المنت است. این پاسخ برگرفته از پاسخ خانم مریم علی کرمی
به شماره دانشجویی ۹۷۳۱۰۴۵ است.

\پایان{پاسخ}
