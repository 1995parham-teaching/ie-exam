\documentclass[../main.tex]{subfiles}

\begin{document}

\قسمت{چرا \متن‌لاتین{HTTP2}؟}

\پاراگراف{}
در نسخه ۱.۱ از پروتکل \متن‌لاتین{HTTP} بحث \متن‌لاتین{Pipelining} مطرح شد، چه چیزی باعث شد که عملا این بحث بهبود کمی در کارایی این پروتکل داشته باشد؟
در \متن‌لاتین{HTTP2}، مساله‌ی \متن‌لاتین{Pipelining} چگونه حل شده است؟

\پاراگراف{}
۲ نمره

\begin{answer}

در \متن‌لاتین{Pipelining} در \متن‌لاتین{HTTP} نسخه ۱.۱ نیاز دارد که ترتیب درخواست‌ها حفظ شود
چرا که امکان متوجه شدن ترتیب بسته‌ها از خود بسته‌ها متوجه شود.
بنابراین مشکل بلاک شدن درخواست‌ها به دلیل پردازش کند یک درخواست
وجود دارد (ذکر این نکته به عنوان دلیل عدم استفاده از \متن‌لاتین{Pipelining} نیمی از نمره را دارد). این مساله در \متن‌لاتین{HTTP2} با استفاده از \متن‌لاتین{Multiplexing} حل شده است.

\end{answer}

\end{document}
