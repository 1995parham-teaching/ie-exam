\documentclass[]{article}

\usepackage{hyperref}
\usepackage{import}
\usepackage{listings}
\usepackage{minted}
\usepackage[dvipsnames]{xcolor}
\usepackage{hyperref}
\usepackage{graphicx}
\usepackage{geometry}
 \geometry{
 a4paper,
 left=2.2cm,
 right=2.2cm,
 top=2.5cm,
 bottom=2.5cm
 }

% \usepackage[disable]{todonotes} % answers not showed
\usepackage[draft]{todonotes}   % answers showed

% based on
\newcommand\answer[2][]{\todo[inline, caption={2do}, #1]{
    \begin{persian}
    \begin{minipage}{\textwidth-4pt}
      پاسخ.
      #2
    \end{minipage}
    \end{persian}
  }
}

\usepackage[localise]{xepersian}
\settextfont{Vazir}
\setlatintextfont{Roboto}

\colorlet{LightGray}{Gray!30!}

\usepackage{subfiles}

\title{آزمون پایان‌ترم‌\\مهندسی اینترنت}
\author{پرهام الوانی}
\date{نیم سال دوم ۱۳۹۹-۱۴۰۰\\زمان آزمون: ۱ ساعت و ۴۰ دقیقه}

\newcommand\Extra{\textcolor{Orange}{امتیازی.}}
\newenvironment{extra}{
  \begin{quote}
  \color{Orange}
  \textbf{امتیازی.}
}{
  \end{quote}
}
\eqenvironment{امتیازی}{extra}

\begin{document}

    \maketitle
    \tableofcontents
    \pagebreak

    \subfile{introduction/main}
    \pagebreak
    \subfile{http2/main}
    \subfile{cookie-2/main}
    \subfile{csrf/main}
    \subfile{cgi/main}
    \subfile{micro-service/main}
    \subfile{aas/main}
    \subfile{css-display/main}
    \subfile{redirect/main}
    \subfile{slice-vs-array/main}
    \vspace*{\fill}
    \begin{center}
این سند برپایه بسته \متن‌لاتین{\زی‌پرشین} گونه \متن‌لاتین{\گونه‌زی‌پرشین} توسعه پیدا کرده است.
    \end{center}

\end{document}
