\documentclass[]{article}

\usepackage{hyperref}
\usepackage{import}
\usepackage{listings}
\usepackage{minted}
\usepackage[dvipsnames]{xcolor}
\usepackage{hyperref}
\usepackage{graphicx}
\usepackage{geometry}
 \geometry{
 a4paper,
 left=2.2cm,
 right=2.2cm,
 top=2.5cm,
 bottom=2.5cm
 }

% \usepackage[disable]{todonotes} % answers not showed
\usepackage[draft]{todonotes}   % answers showed

% based on
\newcommand\answer[2][]{\todo[inline, caption={2do}, #1]{
    \begin{persian}
    \begin{minipage}{\textwidth-4pt}
      پاسخ.
      #2
    \end{minipage}
    \end{persian}
  }
}

\usepackage[localise]{xepersian}
\settextfont{Vazir}
\setlatintextfont{Roboto}

% for listing environments
\colorlet{LightGray}{Gray!30!}

% for href
\eqcommand{تارنما}{href}

\usepackage{subfiles}
\usepackage{parhamExam}

\title{مهندسی اینترنت}
\author{پرهام الوانی}
\date{۲۲ دی ۱۴۰۰}
\university{دانشگاه صنعتی امیرکبیر}
\exam{آزمون پایانی}
\semester{نیم‌سال اول ۱۴۰۰-۱۴۰۱}
\duration{۱۲۰ دقیقه}

\begin{document}
  \maketitle
  \tableofcontents
  \pagebreak

  \subfile{introduction/main}
  \vspace*{\fill}
  \pagebreak
  \subfile{auth/main} % 2
  \subfile{http2/main} % 2
  \subfile{host-header/main} % 1
  \subfile{http-1/main} % 3
  \subfile{tabindex/main} % 1
  \vspace*{\fill}
  \begin{center}
این سند برپایه بسته \متن‌لاتین{\زی‌پرشین} گونه \متن‌لاتین{\گونه‌زی‌پرشین} توسعه پیدا کرده است.
  \end{center}

\end{document}
