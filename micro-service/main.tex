\قسمت{سرویس‌های کوچولو}

\زیرقسمت{تراکنش‌ها}

از شما خواسته شده است یک اپلیکشن مالی طراحی کنید، این اپلیکشن تراکنش‌\پانویس{transaction}های زیادی دارد. تیم شما یک تیم کوچک است، آیا از معماری میکروسرویس استفاده می‌کنید؟

\زیرقسمت{رای‌گیری توزیع شده}

از شما خواسته شده است یک سامانه رای‌گیری طراحی کنید. این سامانه رای‌گیری در همه شهرها مستقر می‌شود و در پایتخت یک سامانه مرکزی دارد که همه داده‌ها را جمع می‌کند. مدت رای‌گیری یک روز کامل است و احتمال خرابی در شبکه بین شهرها وجود دارد. در نظر داشته باشید هیچگاه نباید عملیات رای‌گیری در یک شهر متوقف شود.
طراحی شما به این صورت است که همه شهرها رای‌گیری را انجام می‌دهند و نتایج را به پایتخت به صورت لحظه‌ای ارسال می‌کند. در صورتی که خطا شبکه داشته باشیم نتایج ذخیره شده و بعد از رفع خطا ارسال می‌شوند.

در صورتی که نتایج نهایی اهمیت داشته باشند آیا می‌توان به نتایج سامانه شما اطمینان کرد؟ توضیح دهید. در صورتی که بخواهیم به صورت زنده هم نتایج را داشته باشیم آیا راهی وجود دارد که این نتایج همواره صحیح باشند؟ (راهنمایی: به تئوری \متن‌لاتین{CAP} دقت کنید.)

۳ نمره
